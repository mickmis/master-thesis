% ------------------------------------------------------------------------------------------------------
\subsubsection{Query Step 3:  Data Export}

% intro / problem with pic-sure / overview of what is needed
Using data export with only the native \emph{select} and \emph{where} clauses of the PIC-SURE API reveals a bit limiting as IRCT originally supports only tabular results, i.e. a classic row-columns result.
Moreover there is no support for custom export parameters such as different types of data, only format of the result can be chosen between \emph{TABULAR} or \emph{JSON}.
On the other end, backends such as tranSMART supports elaborated parameters for getting results, which is a required features for our system.
For these reasons we are not using the native export mechanism of PIC-SURE in order to select the format of the data, but are defining a custom \emph{process} clause named \verb|EXPORT|, if the resource supports it.
Modifications in Glowing Bear to use PIC-SURE can be summarized in three points:
\begin{itemize}
    \item Get available data types and file formats for a specific query (export parameters);
    \item Submit export request;
    \item List exports and download them.
\end{itemize}

\paragraph{Available Export Parameters}

% API calls
\verb|ResourceService.getExportFileFormats()| requests to the back end the supported file formats for the exports.
\verb|getExportDataFormats()| requests to the back end the supported types of data to export, given the current query.
These two requests are merged into \verb|getExportFormats()| and the API calls are replaced by using a PIC-SURE query with specific \emph{select} clauses using the operation \verb|EXPORT|, example:
\begin{verbatim}
POST /queryService/runQuery
{
  "select": [ {
    "operation": "EXPORT",
    "fields": { "SUPPORTED_FORMATS": "[data, file]" }
  }, 
    <query_select_clauses>
  ],
  "where": [ <query_where_clauses> ],
  "alias": "<query_alias>"
} 
\end{verbatim}

% in service
\verb|QueryService.updateExports()| is modified to take into account the merging of these two methods and the result that is now represented differently.

\paragraph{Export Request}

% API calls
\verb|ResourceService.createExportJob()| and \verb|runExportJob()| that respectively create an run an export job are merged into \verb|runExportJob()|.
The call made is similar the previous, using \emph{select} clauses to specify the parameters of the export job:
\begin{verbatim}
POST /queryService/runQuery
{
  "select": [ {
    "operation": "EXPORT",
    "fields": { "data": "clinical", "file": "TSV" }
  }, {
    "operation": "EXPORT",
    "fields": { "data": "mrna", "file": "JSON" }
  }, 
    <query_select_clauses>
  ],
  "where": [ <query_where_clauses> ],
  "alias": "<query_alias>"
} 
\end{verbatim}

Note that the name of the fields match the possible values of the \verb|SUPPORTED_FORMATS| field.

% in component
\verb|GbExportComponent.runExportJob()| is modified to take into account the merging of these two methods and the result that is now represented differently.


\paragraph{Manage Exports}

% list
\verb|GbExportComponent.updateExportJobs()| takes care of refreshing the list of exports previously made by using \verb|ResourceService.getExportJobs()| to make the API request.
This API request is modified to:
\begin{verbatim}
GET /resultService/available
\end{verbatim}

% download
\verb|GbExportComponent.downloadExportJob()| take care handling the download of the file containing the exported results, and to do so it uses the API request made in \verb|ResourceService.downloadExportJob()|.
This call is modified to:
\begin{verbatim}
GET /resultService/result/<result_id>/ZIP?download=yes
\end{verbatim}

Note that before that a first call is made to get the available formats for the results, but in the case of multi-dimensional data it will always return \verb|ZIP|:
\begin{verbatim}
GET /resultService/availableFormats/<result_id>
\end{verbatim}

% todo: cancel / archive not covered
% todo: data table, using param in select EXPORT? process?


% ------------------------------------------------------------------------------------------------------

\subsubsection{Saving a Query}
On top of modifying the \verb|Query| model to fit the PIC-SURE format, only the API calls to handle query saving need to be modified, the calling code remains the same.
The IRCT implementation to save and re-use saved queries is not complete: it is originally possible to only save queries, but not list or load them. 
See section~\ref{sec:irctsavedqueries} for the additional implementation in IRCT to add these features.

% saveQuery
In \verb|ResourceService.saveQuery()| the API request becomes:
\begin{verbatim}
POST /queryService/queries
{
    "queryName": <query_name>,
    "query": {
        <query_body>
    }
}    
\end{verbatim}

Response:
\begin{verbatim}
{
    "queryId": <query_id>
}    
\end{verbatim}

% getQueries
In  \verb|ResourceService.getQueries()| the API request becomes:
\begin{verbatim}
GET /queryService/queries
\end{verbatim}

Response:
\begin{verbatim}
{
    [
        "queryId": <query_id>,
        "queryName": <query_name>,
        "query": {
            <query_body>
        }
    ], [
        ...
    ],
    ...
}    
\end{verbatim}

% updateQuery
In  \verb|ResourceService.updateQuery()| the API request becomes:
\begin{verbatim}
PUT /queryService/queries/<query_id>
{
    "queryName": <query_name>,
    "query": {
        <query_body>
    }
}
\end{verbatim}

Response:
\begin{verbatim}
{
    "message": <status_message>
} 
\end{verbatim}

% deleteQuery
In  \verb|ResourceService.deleteQuery()| the API request becomes:
\begin{verbatim}
DELETE /queryService/queries/<query_id>
\end{verbatim}

Response:
\begin{verbatim}
{
    "message": <status_message>
} 
\end{verbatim}

% todo: restoreQuery()

% ------------------------------------------------------------------------------------------------------

\paragraph{Saving Queries: Additional API Calls}
\label{sec:irctsavedqueries}

% rest api
The original IRCT does implement an API call to save a specific query: \verb|POST /queryService/savequery|.
However it does not provide a way to manage the existing queries.
As this is a required feature, we are adding it in the IRCT core.
We modify the class \verb|edu.harvard.hms.dbmi.bd2k.irct.cl.rest.QueryService| to add the following methods:
\begin{itemize}
    \item \verb|getQueries()|, implementing API call \verb|GET /queryService/queries|
    \item \verb|saveQuery()|, implementing API call \verb|POST /queryService/queries|
    \item \verb|deleteQuery()|, implementing API call \verb|DELETE /queryService/queries/<query_id>|
    \item \verb|updateQuery()|, implementing API call \verb|PUT /queryService/queries/<query_id>|
\end{itemize}

The mapping API calls to methods is done using JAX-RS~\cite{wiki:jaxrs}.

% controller
The class \verb|irct.controller.QueryController| that implements the logic of managing queries.
It needs to be modified so that the saved queries are associated to an user: the users should be able to manage only their own queries.

\paragraph{Data Export: Support of Additional Data Type}

% why + add type
The original IRCT supports results of following types: \verb|TABULAR|, \verb|JSON|, \verb|HTML|, \verb|IMAGE|.
Supports for types of results is represented through the enum \verb|edu.harvard.hms.dbmi.bd2k.irct.model.result.ResultDataType| and the the code that manipulates it.
The data exports in tranSMART and i2b2 gives multi-dimensional data that are not supported with this construction.
To support such results, we add an IRCT result type \verb|ZIP|.
The resources using this type declare it through their \verb|irct.model.resource.implementation.| \\
\verb|QueryResourceImplementationInterface.getQueryDataType()| method.

% things to modify as consequence for added type
In the newly created package \verb|irct.model.result.zip| we create a new class that implement the \verb|ZIP| result type, to do do it implements the classes
\verb|irct.model.result.Data| and \\
\verb|irct.model.result.Persistable|.
This is a pretty simple implementation that only stores the resulting files from resources on the file system.

% ------------------------------------------------------------------------------------------------------

\paragraph{SHRINE}
The SHRINE API is basically a limited i2b2 API (saves for some very minor additional fields).
It supports only counts, which means no data exports.
The only significant difference is in the results, as one query brings several results.
We take the i2b2 implementation and slightly modify to the purpose of being compatible with SHRINE nodes.

The resource exposes the following \emph{select} operations:
\begin{itemize}
    \item \verb|AGGREGATE|, fields:
    \begin{itemize}
        \item \verb|FUNCTION|, mandatory, possible values:
        \verb|COUNT|
        
        \item \verb|DIMENSION|, optional, possible values:
        \verb|observation|,
        \verb|patient|
    \end{itemize}
\end{itemize}

And the following \emph{where} predicates:
\begin{itemize}
    \item \verb|CONSTRAIN_CONCEPT|, constraint based on a concept, field:
    \verb|OPERATOR|, optional, possible values:
    \begin{itemize}
        \item numeric values: \verb|==|, \verb|<=|, \verb|>=|, \verb|<|, \verb|>|
        \item string values: \verb|LIKE[exact]|, \verb|LIKE[begin]|, \verb|LIKE[end]|, \verb|LIKE[contains]|
    \end{itemize}
\end{itemize}

% ------------------------------------------------------------------------------------------------------

\subsubsection{Query Step 2: Data Selection}

% overview
In the PIC-SURE paradigm, this step is about preparing the \emph{select} statement of the query: the output for step 3 is the added \emph{select} causes to the query.
The changes in this part are minimal, as they are mainly about the modification of the resulting JSON containing the concepts the user wishes to get as a result, but they represent the same information.

% modifications
Originally this information is stored in a \verb|CombinationConstraint|, with individual constraints linked by a \verb|OR|.
Here for this purpose we are creating a new \verb|PICSURESelectAttribute| model containing the different \emph{select} clauses, which for each clause holds the path of the concept and the associated data type.
For the sake of consistency the method \verb|ConstraintService.generateProjectionConstraint()| is renamed to \verb|generateSelectAttribute()| and returns a \verb|PICSURESelectAttribute|.
Some additional minor modifications are made in the methods using this, such as \verb|QueryService.updateCounts_2()|, \verb|updateExports()| and \verb|GbExportComponent.runExportJob()|.

% example
Example of \emph{select} clauses generated during the second step:
\begin{verbatim}
"select": [ {
    "field": {
        "pui": "/<resource>/Public Studies/CATEGORICAL_VALUES/Demography/Age/",
        "dataType": "INTEGER"
    }
    }, {
    "field": {
        "pui": "/<resource>/Public Studies/CATEGORICAL_VALUES/Demography/Gender/Male/",
        "dataType": "ENUM_VALUE"
    }
    }, {
    "field": {
        "pui": "/<resource>/Public Studies/CATEGORICAL_VALUES/Demography/Gender/Female/",
        "dataType": "ENUM_VALUE"
    }
} ]
\end{verbatim}

% ------------------------------------------------------------------------------------------------------



\subsubsection{Authentication \& Authorization}

IRCT uses OpenID Connect to authenticate and authorize the users of its client applications, while Glowing Bear originally implements the client side of the OAuth2 protocol for authorization with tranSMART. 
The modification to Glowing Bear is the migration from the authorization protocol OAuth2~\cite{oauth2} to the authentication and authorization protocol OpenID Connect~\cite{openidconnect}.
Since OpenID Connect is a layer on top of OAuth2, the modifications to migrate the client code from OAuth2 to OpenID Connect are not significant, and are made by making use of the \emph{angular-auth-oidc-client} library~\cite{angular-auth-oidc-client}.
The flow remains the same: Glowing Bear obtains a JSON Web Token (JWT) from the OpenID Connect provider, which will then be embedded in the header of the HTTP requests to IRCT.
These modifications are made by replacing the \verb|EndPointService| by \verb|AuthenticationService|, which implements the logic of the authentication.
Additionally a small component \verb|AutoLoginComponent| redirects to the login page of the OIDC server if the authentication token is not valid or not present.
% ------------------------------------------------------------------------------------------------------

\subparagraph{Pedigree Constraint}
Constraints based on pedigree are special in that they are based on other constraints (e.g. getting the parents of patients aged more than 50).
We add for them a predicate \verb|CONSTRAINT_PEDIGREE|, which has three fields:
\begin{itemize}
    \item \verb|TYPE|, required: list of permitted values which are either the type of pedigree (parents of, siblings of, etc.) or the indicator of the end of the pedigree constraints \verb|CONSTRAINT_END| (see example after)
    \item \verb|BIOLOGICAL|, optional, default to \verb|BOTH|: either \verb|YES|, \verb|NO| or \verb|BOTH|
    \item \verb|SHARE_HOUSEHOLD|, optional, default to \verb|BOTH|: either \verb|YES|, \verb|NO| or \verb|BOTH|
\end{itemize}

Example of a pedigree \emph{where} clause:
\begin{verbatim}
{
    "where": [
        {"predicate": "CONSTRAINT_PEDIGREE", "fields": { 
          "TYPE": "PARENTS_OF", "BIOLOGICAL":"YES" 
        } },
        <classic constraints>..., 
        {"predicate": "CONSTRAINT_PEDIGREE", "fields": { "TYPE": "CONSTRAINT_END" } }
    ]
}
\end{verbatim}

Similarly to other constraints, the logic construction is implemented into \\
\verb|GbPedigreeConstraintComponent| and the model representing such a constraint is \verb|PedigreeConstraint|.
It is only enabled if \verb|IRCTResourceService.supportsPedigree()| determines so.

% ------------------------------------------------------------------------------------------------------

\subparagraph{tranSMART Studies}

% general + into tree
The tranSMART REST API v2 has several calls that are specific to studies, while the PIC-SURE API does not have a direct equivalent to this.
However since the studies are completely embedded within the concept tree, i.e. a concept belongs to one study or it is cross-study, they can simply be abstracted into the tree of entities exposed by PIC-SURE: we create a \verb|STUDY| data type for the tree entities (see~\ref{sec:gb-tree}).
\verb|TreeNodeService.isTreeNodeAStudy()| implements the recognition of the \verb|STUDY| data type.
%todo: in model

% list of studies
If the resource supports it, \verb|ConstraintService.loadStudies()| loads the list of studies by calling \verb|ResourceService.getStudies()|, which is modified to use the following PIC-SURE call:
\begin{verbatim}
POST /queryService/runQuery
{
  "select": [ {
    "operation": "AGGREGATE",
    "fields": {
        "FUNCTION": "values",
        "DIMENSION": "study"
    }
  } ],
  "alias": "get_studies"
}
\end{verbatim}
% ------------------------------------------------------------------------------------------------------


\paragraph{Specific Features}
\label{sec:specific-features}

\verb|IRCTResourceService| provides methods that determine if some specific features are supported, which allows some features of Glowing Bear to be enabled or disabled:
\begin{itemize}
    \item \verb|supportsCounts()|: returns \emph{true} if queries of type \verb|SELECT AGGREGATE COUNT| are supported, which enables the live counts display in the UI
    \item \verb|supportedCountsDimensions()|: if \verb|supportsCounts()| is \emph{true}, returns the countable dimensions (most commonly \emph{Patients} and \emph{Observations}).
    \item \verb|supportsNestedClauses()|: returns \emph{true} if the \emph{where} clause predicates \verb|CLAUSE_NEST| and \\
    \verb|CLAUSE_UNNEST| allowing nesting are supported
    \item \verb|supportsMinMax()| returns \emph{true} if queries of type \verb|SELECT AGGREGATE MIN/MAX| are supported
    \item \verb|supportsStudies()| / \verb|supportsClinicalTrials()|: returns \emph{true} if queries based on those specific dimensions are supported 
    \item \verb|supportsPedigrees()|: returns \emph{true} if constraints based on pedigrees are supported
    \item \verb|supportsGetTreeWithDepth()|: returns \emph{true} if the tree can be retrieved with a specified depth (if the resource declare a tree relationship as \verb|CHILDREN-DEPTH-X|)
    \item \verb|supportsDataExport()|: returns \emph{true} if data export queries are supported
\end{itemize}

% ------------------------------------------------------------------------------------------------------

\newpage
\section{Implementation of the Interoperability Layer}
% Here is described component-per-component the implementation of the solution presented in the last section. 
% Each of the sections first explains what is the current status of the relevant components, and then what needs to be modified or implemented.
% First we describe the deployment of all the relevant components that need to be used in the system (refer to figure~\ref{fig:sysdiagramobj1}).
% Then we are going over all the modifications made to Glowing Bear, regarding the change of the API it uses.
% Finally we go over the modifications of the core of IRCT, and the implementation of all the resources it uses.
% ------------------------------------------------------------------------------------------------------

\subparagraph{Patient Set Constraint}
Through the UI, the user of Glowing Bear can import a patient set and use it as a constraint, this handled by \verb|GbPatientSetConstraintComponent| and stored into a model \verb|PatientSetConstraint|, which are modified to adapt to the PIC-SURE format.
It supports either an array of patient identifiers, or the identifier of a patient set.

Example of patient set \emph{where} clauses:
\begin{verbatim}
{
    "where": [
        {"predicate": "CONSTRAINT_PATIENT_SET", "fields": { "PATIENT_SET_ID": "52" } },
        {"predicate": "CONSTRAINT_PATIENT_SET", "fields": { 
          "PATIENT_IDS": "[2, 32, 96]" 
        } }
    ]
}
\end{verbatim}
% ------------------------------------------------------------------------------------------------------

\paragraph{tranSMART 17.1}
\label{sec:irct-res-transmart-17.1}

We make the tranSMART 17.1 resource declare the following \emph{select} operations:
\begin{itemize}
    \item \verb|AGGREGATE|, fields:
    \begin{itemize}
        \item \verb|FUNCTION|, mandatory, possible values:
        \verb|COUNT|,
        \verb|MIN|,
        \verb|MAX|,
        \verb|VALUES|
        
        \item \verb|DIMENSION|, optional, possible values:
        \verb|observation|,
        \verb|patient|,
        \verb|study|,
        \verb|trial_visit|
    \end{itemize}
    
    \item \verb|EXPORT|, fields:
    \begin{itemize}
        \item \verb|SUPPORTED_FORMATS|: possible values: \verb|file|, \verb|data|
        \item \verb|file|: value among the ones returns with \verb|SUPPORTED_FORMAT|
    \end{itemize}
    
    \item (without predicate): select data to export
\end{itemize}

And the following \emph{where} predicates:
\begin{itemize}
    \item \verb|CONSTRAIN_CONCEPT|, constraint based on a concept, field:
    \verb|OPERATOR|, optional, possible values:
    \begin{itemize}
        \item numeric values: \verb|==|, \verb|<=|, \verb|>=|, \verb|<|, \verb|>|
        \item string values: \verb|LIKE[exact]|, \verb|LIKE[begin]|, \verb|LIKE[end]|, \verb|LIKE[contains]|
    \end{itemize}

    \item \verb|CONSTRAIN_FIELD|, constraint based on a field in one of the dimension, fields:
    \begin{itemize}
        \item \verb|DIMENSION|
        \item \verb|FIELD|
        \item \verb|OPERATOR|
        \item \verb|VALUE|
    \end{itemize}
    
    \item \verb|CONSTRAIN_PEDIGREE|, constraint based on relationship between patients, fields:
    \begin{itemize}
        \item \verb|TYPE|, e.g. \verb|PARENTS_OF|, etc. among permitted values
        \item \verb|BIOLOGICAL|, possible values: \verb|YES|, \verb|NO|, \verb|BOTH|
        \item \verb|CONSTRAINT_END|
    \end{itemize}
    
    \item \verb|CONSTRAINT_PATIENT_SET|, constraint based on a patient set, fields:
    \begin{itemize}
        \item \verb|PATIENT_SET_ID|: identifier of a patient set stored in the back end
        \item \verb|PATIENT_IDS|: array of patient identifiers
    \end{itemize}
    
    \item \verb|NESTING|, fields: \verb|TYPE| (values: \verb|start| or \verb|end|)
    
\end{itemize}

% lib
First in order to support all the calls to tranSMART 17.1 backends that are needed, a simple client library for tranSMART REST API v2 is developed.
In terms of features, this library offers all calls that were previously available in Glowing Bear, and that will be used to answer the calls from the PIC-SURE API.

% tree
The tree exposed through the PIC-SURE API is the same as the tranSMART 17.1.
Natively PIC-SURE does not support browsing the tree at more than one level at a time, so we add another relationship supported in browsing the tree: \verb|CHILDREN-DEPTH-X|, that will return children with a depth up to \verb|X|.
The visual attributes parameters is added in order to display with the proper type in the UI the node, an example of this is the study: it is a \verb|STUDY| node in the tree, and in order to be displayed as such in the UI the visual attributes parameters is set as such.

% queries: select / where
All the predicates and operations described before are translated into the native API in a straightforward way, as all the information for input is available.
The is a special case: if the queried concept belongs to a study, a study constraint is added, while if the concept is cross-study it is left as is.
The nesting of queries described section~\ref{sec:gb-logical-operators} is implemented in the resource and translated in the native API.

%todo: construction of query: need to fetch in tree the constraint to be used

%todo: \paragraph{tranSMART 16.2} it is not tranSMART 16.2 :( but i2b2 transmart

% ------------------------------------------------------------------------------------------------------

We make the i2b2 resource declare the following \emph{select} operations:
\begin{itemize}
    \item \verb|AGGREGATE|, fields:
    \begin{itemize}
        \item \verb|FUNCTION|, mandatory, possible values:
        \verb|COUNT|
        
        \item \verb|DIMENSION|, optional, possible values:
        \verb|observation|,
        \verb|patient|
    \end{itemize}
    
    \item \verb|EXPORT|, fields:
    \begin{itemize}
        \item \verb|SUPPORTED_FORMATS|: possible values: \verb|file| %, \verb|data|
        \item \verb|file|: value among the ones returns with \verb|SUPPORTED_FORMAT|
    \end{itemize}
    
    \item (without predicate): select data to export
\end{itemize}

And the following \emph{where} predicates:
\begin{itemize}
    \item \verb|CONSTRAIN_CONCEPT|, constraint based on a concept, field:
    \verb|OPERATOR|, optional, possible values:
    \begin{itemize}
        \item numeric values: \verb|==|, \verb|<=|, \verb|>=|, \verb|<|, \verb|>|
        \item string values: \verb|LIKE[exact]|, \verb|LIKE[begin]|, \verb|LIKE[end]|, \verb|LIKE[contains]|
    \end{itemize}
    
    \item \verb|CONSTRAINT_PATIENT_SET|, constraint based on a patient set, fields:
    \begin{itemize}
        \item \verb|PATIENT_SET_ID|: identifier of a patient set stored in the back end
        \item \verb|PATIENT_IDS|: array of patient identifiers
    \end{itemize}
\end{itemize}


% ------------------------------------------------------------------------------------------------------
