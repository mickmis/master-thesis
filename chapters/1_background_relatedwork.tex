\chapter{Background \& Related Work}
% todo: maybe integrate the sections in the introduction

\section{Background}
% contains purely descriptive information
This section offers some important background information providing context to what is presented in the later chapters. 
First an overview of the different open-source clinical research systems is offered, 
after which the different front ends allowing to exploit them are presented.

\subsection{Clinical Research Systems}

talk about the inner data model of each


\subsubsection{tranSMART}
%TODO

\subsubsection{i2b2}
Informatics for Integrating Biology and the Bedside (i2b2) is a NIH-funded National Center for Biomedical Computing (NCBC) that is developing a software that goes by the same name. 
%todo



%TODO
% core server, demo data
% db compat: 3

\subsubsection{SHRINE}
%TODO

\subsubsection{PIC-SURE API / IRCT}

% overview
The National Institutes of Health (NIH) of the U.S. government launched in 2013 the first phase of the Big Data to Knowledge (BD2K)~\cite{BD2K} whose one of the goal is to exploit the immense amount of big data information to advance knowledge in modern biomedical research.
One of those center of excellence is the PIC-SURE (Patient-centered Information Commons: Standardized Unification of Research Elements)~\cite{PIC-SURE} whose goal is create a scalable toolkit to enable patient-centered information commons.
etc. created within HMS DBMI [cite], open source, 
led by paul avillach One of their achievement is the BD2K PIC-SURE RESTful API [bd2k-picsure.hms.harvard.edu] that aims to incorporate multiple heterogeneous clinical research systems. 
The official implementation of this API is called Inter Resource Communication Tool (IRCT).

% pic-sure api description
While not actually being a clinical research system, the IRCT can be seen as a meta system that expose data of other systems through a common API.
Restful api for heteregeonous datasets ,decentralized fashion, access with single comm. Layer -> interoperability layer (check the api doc pdf)

% IRCT components
The IRCT implementation is the combination of four different components, that are open-source and available on~\cite{IRCT-github}. They are implemented in Java and use standard technologies: web application archive (WAR)~\cite{wiki:war} for deployment, Hibernate~\cite{wiki:hibernate} for data storage.
First the Communication Layer (IRCT-CL) implements the RESTful service that is exposed and documented by~\cite{PIC-SURE-API}. 
The core component is the Application Programming Interface (IRCT-API), it handles the execution of queries and processing of results.
An instance can be extended using the IRCT Extension (IRCT-EXT) that provides hooks and additional features without having to modify the core code.
Finally the Resource Interface (IRCT-RI) connects to the different resources through connectors.


%https://www.nature.com/articles/sdata201696
%http://dbmi.hms.harvard.edu/news/datathon-hackathon-sep-14-15
%info about hackaton: dana farber is looking for i2b2 frontend replacement 

% notes on irct 
%PREDICATE MANDATORY in query
%DATA type is needed in query

\paragraph{Technical Background}
PIC-SURE is resource based: each source of data (e.g. i2b2, tranSMART, etc.) is considered a resource.
Each of these resource declare through their configuration the kind of clauses they support. 
The configuration is done through the database.
Four types of clauses are supported: \emph{select}, \emph{where}, \emph{process} and \emph{join}, but here we are mainly making use of \emph{select} and \emph{where}.

% select
The \emph{select} clauses specify the data that the user of the API wants extracted from the database.
The resources declare what kind of operations their \emph{select} support, for example \emph{AGGREGATE} can be defined to extract aggregated values.
Then each of the operations (or the default operation when none is declared), declare the fields they support.
For \emph{AGGREGATE}, the resource could declare two fields:
\begin{itemize}
    \item \emph{FUNCTION}: what function to use among a list of permitted values, e.g. \emph{COUNT}, \emph{MIN}, \emph{MAX}
    \item \emph{DIMENSION}: the dimension along which to do to the aggregation, e.g. \emph{PATIENT}
\end{itemize}
Each of the fields declare the type of value they take: either an enumerated value, or a data type such as \emph{String} or \emph{Integer} that has to specify the Java class implementing it.
Example of two \emph{select} clauses:
\begin{verbatim}
"select": [ {
    "operation": "AGGREGATE",
    "fields": {
        "FUNCTION": "COUNT",
        "DIMENSION": "PATIENT"
    }
}, {
    "field": {
        "pui": "/resource/study/Age/",
        "dataType": "INTEGER"
    }
} ]
\end{verbatim}

% where 
The other main clause supported is \emph{where} and is used to specified constraints on the data to be selected.
Just like for \emph{select} and the supported operations, the resource declares the predicates that \emph{where} supports.
The predicates can have fields, and fields have a type: this is just like \emph{select}. 
Example:
\begin{verbatim}
"where": [ {
    "field": {
        "pui": ""/resource/study/Age/"",
        "dataType": "INTEGER"
    },
    "predicate": "CONSTRAIN_VALUE_NUMERIC",
    "fields": {
        "OPERATOR": ">=",
        "CONSTRAINT": "20"
    }
} ]
\end{verbatim}

% tree
In order to construct queries made of the clauses previously described, the resources expose a tree of entities.
Each of those entities, if they are queryable, declare a data type.
Each of the predicates used for the \emph{where} clauses also declare one or more supported data types: this is the mechanism used to know which entities support which predicates.

\subsection{Cohort Exploration Front Ends}
\subsubsection{Glowing Bear}
%§ GB: Most recent Angular version is now ng5, AngularJS refers to the first generation of Angular which is sth. glowing bear no longer uses. see https://blog.angular.io/version-5-0-0-of-angular-now-available-37e414935ced

\subsubsection{tranSMARTApp}
%are there other transmart clients that exist?


\subsubsection{i2b2 Clients}
\paragraph{Webclient}

\paragraph{Workbench}

\section{Related Work}
% say if open source or not 
% this is more analytic than background: offering an overview of what else is done and compare it to what we are doing, reuse what was said in the background
%CAVA: http://perer.org/papers/adamPerer-CAVA-IVS2014.pdf

%all the work mentioned in background

%they had a partnership with pic sure: %https://academic.oup.com/jamia/article/22/6/1132/2357622?searchresult=1
%pic sure 'wow' story (nhanes): %https://bd2kccc.org/wp-content/uploads/2017/02/15_PIC-SURE_FInal.pdf

% todo: OMOP: https://www.ohdsi.org/data-standardization/the-common-data-model/
% -> explore it
% have a list of data models (perpendicular to list of back end systems)