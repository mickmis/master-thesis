% todo: have disambiguation if needed
% vocabulary
% query terms: concepts (i2b2) / constraints (more than that)
% query: constraint


\chapter{Introduction}
% a summary of the introduction chapter
% todo: do not number any section and chapter here?

\begin{itemize}
    \item motivation exposes problems
    \item problems to be solved lead to objectives
    \item objectives translated into technical requirements
    \item technical requirements met with architecture solution
    \item solution designed and implemented
    \item evaluation of how implementation meet requirements
\end{itemize}

\section{Problem Statement}
\label{sec:problem}

\subsection*{Motivation}

\begin{itemize}
    \item precision medicine: what is it and why it's good and we need it
    \item it requires to handle large and heterogeneous data (clinical + genomic)
    \item the more data exploited, better it is for it
\end{itemize}

%Being able to exploit large and heterogeneous clinical and genomic data is crucial for realizing the promise of  precision medicine to its full potential. Yet, due to the current presence of multiple and fragmented systems at different clinical sites, it is often extremely difficult to enable researchers to access the data they need. Privacy and security concerns also represent major obstacles that need to be overcome in order to get access to sensitive medical data that are usually not exposed by clinical sites for the fear of data leaks. 
% todo: get inspiration from papers
% 2 problems (here)  --> 2 objectives (in objective)

% IMPORTANT: why do these 2 objectives should be taken together, and not separately, and why not others

\subsection*{Objectives}
% check http://www.ceptara.com/blog/how-to-write-problem-statement
%-Why this is a hard/open problem?
%-State-of-the-Art not sufficient (refer to related work?)
% goal: solving this problem: goal: common - privacy preserving

% facilitating adoption / meaningful things -> open-source / docker / secure / etc (overlap with requirements? or should that go to the requirements? check how to split)
% 2 problems (from motivations)  --> 2 objectives (here)


% These characteristics define two main objectives to fulfill. 
% First, the front end in question will need to be able to communicate with the main open-source several clinical research platforms, to cover a major part of the platforms used by researchers.
% Second, the front end , namely MedCo, the only privacy-preserving platform.
% todo: in the analyses of what are the platforms, mention what they are

% --> focus on cohort exploration (why?)


\begin{itemize}
    \item data scattered in several ways / all the constraints that prevent what said previously to be achieved
    \item physically in different institutions / organizations
    \item technologically heterogeneous, (backend systems, apis to access, semantic different, etc.)
    \item legal constraints, risk of leaks
    \item  patients reluctant to share their data
\end{itemize}

% todo: emphasize, bold or italic, the objectives
2 obj that stem from Enable scientists to access more data from a common interface by...
\begin{itemize}
    \item common system to unify different system, have a common one
    \item accessing sensitive data that would otherwise not be shared
\end{itemize}


\section{Solution Requirements}
\label{sec:requirements}

% requirements list
We translate the objectives stated in~\ref{sec:problem} into the following requirements.
Our solution must:

\begin{enumerate}
    \item be compatible with the major open-source clinical research systems % todo: and do what exactly? cohort exploraiton blabla, goal is not just to be compatible
    \item enable sharing of sensitive data in a privacy-preserving way
    \item be free of technical constraint against its use, by:
    \begin{enumerate}
        \item being easy to deploy, even in existing environments
        \item not degrading the experience of existing systems
        \item being secure
        \item being open-source
        \item having a practical runtime
        \item being future-proof
    \end{enumerate}
\end{enumerate}

% why requirements are fulfilling objectives

% reasons for open-source:
% In the open-source spirit, existing technologies and standard. Maximizing the value of the work. 
% Integrating with existing systems.


\section{Solution Overview}

\begin{itemize}
    \item big picture of the whole system in short
    \item how it helps solve the 2 objectives in short
\end{itemize}


\section{Contributions}

\begin{itemize}
    \item list of contributions made
    \item evaluation shows how the contributions are successful
\end{itemize}

% - what we do is a step in the direction of blabla
% convergence of blabla

\section{Outline}
TBD
