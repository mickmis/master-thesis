% todo: have disambiguation if needed
% vocabulary
% query terms: concepts (i2b2) / constraints (more than that) / tree node
% query term type
% query: constraints
% data sources (no pic-sure resource)
% "paper" rather than thesis or document
% ============================

% flow of the paper:
% \begin{itemize}
%     \item motivation exposes problems
%     \item problems to be solved lead to objectives
%     \item objectives translated into technical requirements
%     \item technical requirements met with architecture solution
%     \item solution designed and implemented
%     \item evaluation of how implementation meet requirements
% \end{itemize}

\chapter{Introduction}
% todo: add citation in here
% 1: https://www.karger.com/Article/FullText/481682 : cite

% precision medicine what it needs
Realization of the promise of precision medicine requires large and diversified clinical and genomic data sets for researchers to exploit~\cite{todo1}.
Existing technical solutions to exploit those data sets exist and are widely used, the two main open-source players being i2b2 and tranSMART.

% data scattered in systems
However those clinical research platforms are generally deployed at single clinical sites, or at best a group of them, hosting their own data.
This leads to the presence of multiple and fragmented systems at different locations all over the world, a situation that is not close to change as a lot of effort and money has been invested in those by clinical sites on often tight budgets.
Due to this scattered data, researchers find themselves limited by the amount of data that their own clinical sites produce.

% data sharing
A straightforward solution to this problem is to share the data between the clinical sites, however this faces two challenges.
First challenge is technical, the different clinical research platforms are not interoperable together.
Second is legal and ethic, as medical data are highly sensitive information about individuals.
Those are protected by strict regulations such as HIPAA in the US or GDPR in EU, and data leaks could be devastating to the concerned persons, thus privacy and security guarantees are capital.

% solution to this \ contribution
In this paper we build a system that takes a step towards the technical convergence of those systems by providing an interoperability layer for i2b2 and tranSMART, exploitable by a unique front end, Glowing Bear.
Moreover this interoperability layer offers as well the ability to securely share medical data between clinical sites using the privacy-preserving system MedCo.


\section{Problem Statement}
\label{sec:problem}

\subsection*{Motivation}


% todo: get inspiration from papers
% 2 problems (here)  --> 2 objectives (in objective)

% IMPORTANT: why do these 2 objectives should be taken together, and not separately, and why not others

\subsection*{Objectives}
% check http://www.ceptara.com/blog/how-to-write-problem-statement
%-Why this is a hard/open problem?
%-State-of-the-Art not sufficient (refer to related work?)
% goal: solving this problem: goal: common - privacy preserving

% facilitating adoption / meaningful things -> open-source / docker / secure / etc (overlap with requirements? or should that go to the requirements? check how to split)
% 2 problems (from motivations)  --> 2 objectives (here)


% These characteristics define two main objectives to fulfill. 
% First, the front end in question will need to be able to communicate with the main open-source several clinical research platforms, to cover a major part of the platforms used by researchers.
% Second, the front end , namely MedCo, the only privacy-preserving platform.
% todo: in the analyses of what are the platforms, mention what they are

% --> focus on cohort exploration (why?)


\begin{itemize}
    \item data scattered in several ways / all the constraints that prevent what said previously to be achieved
    \item physically in different institutions / organizations
    \item technologically heterogeneous, (backend systems, apis to access, semantic different, etc.)
    \item legal constraints, risk of leaks
    \item  patients reluctant to share their data
\end{itemize}

% todo: emphasize, bold or italic, the objectives
% 2 obj that stem from Enable scientists to access more data from a common interface by...
\begin{itemize}
    \item common system to unify different system, have a common one
    \item accessing sensitive data that would otherwise not be shared
\end{itemize}

% FROM WARD
% ======== motivation exposes problem


Combining data from these different data warehouses is hard because of technical and interoperability reasons. In this paper we will focus on the first problem. Moving all data to one data warehouse is not feasible, because of the amount of work to do so and because different data warehouses are specialized in different data types (structured clinical data, large scale genomics data, etc.).

Also, although there are many different data warehouses for health care data, many of them need similar functionality: patient cohort selection, data selection, exploratory analysis, export functionality. However, different data warehouses currently waste financial and development resources on building and maintaining their user interfaces.

% ======== problems solved to objectives 
In this paper we aim to:
- allow one existing, modern data warehouse user interface, with an interoperability layer, to run on multiple data warehouses, specifically the i2b2, tranSMART and medco data warehouses.
- lay the basis for allowing multiple specialized data sources to connect to one user interface, allowing the user to combine information from patients spread over them.


\section{Solution Requirements}
\label{sec:requirements}

% requirements list
We translate the objectives stated in~\ref{sec:problem} into the following requirements.
Our solution must:

\begin{enumerate}
    % the systems ARE a requirement
    \item be compatible with the two major open-source clinical research systems: tranSMART (v17.1) and i2b2  % todo: and do what exactly? cohort exploraiton blabla, goal is not just to be compatible
    \item enable sharing of sensitive data in a privacy-preserving way: with MedCo
    \item be free of technical constraint against its use, by:
    \begin{enumerate}
        \item being easy to deploy, even in existing environments
        \item not degrading the experience of existing systems
        \item being secure
        \item being open-source
        \item having a practical runtime
        \item being future-proof
    \end{enumerate}
\end{enumerate}

% why requirements are fulfilling objectives

% reasons for open-source:
% In the open-source spirit, existing technologies and standard. Maximizing the value of the work. 
% Integrating with existing systems.

from ward:
% ======== objectives into technical requirements
Interoperability layer
Medco, tranSMART, i2b2
Support for multiple data sources (in the basis)

\section{Solution Overview}

\begin{itemize}
    \item big picture of the whole system in short
    \item how it helps solve the 2 objectives in short
\end{itemize}


\section{Contributions}

\begin{itemize}
    \item list of contributions made
    \item evaluation shows how the contributions are successful
\end{itemize}

% - what we do is a step in the direction of blabla
% convergence of blabla

\section{Outline}
TBD
