\chapter{System Evaluation}
\label{sec:evaluation}

In this section we show how the requirements enumerated section~\ref{sec:requirements} are fulfilled.

\begin{enumerate}
    \item \emph{offer a modern web-based front end for clinical research platforms that allows cohort exploration, based on several types of criteria} \\
    As a first goal we set out to make sure we would use a modern front end, that would allow for cohort exploration. The chosen front end, Glowing Bear, is a user-friendly web application built using modern technologies (Typescript and Angular).
    It offers cohort exploration based on the targeted criteria: inclusion and exclusion, presence of ontology query term, and constraint based on their value.
    
    \item \emph{be compatible with the two major open-source clinical research systems: tranSMART (v17.1) and i2b2} \\
    It supports both targeted platforms i2b2 and tranSMART 17.1.
    It has the acceptable limitation that queries cannot combine terms from different data sources.

    \item \emph{enable sharing of sensitive data in a privacy-preserving way with MedCo} \\
    Its support of MedCo enables users access to sensitive data that would not be shared without the strong privacy and security guarantees offered.

    \item \emph{be easily extensible for future support of additional platforms} \\
    Integration of the PIC-SURE API through IRCT makes our solution offer a framework for future developers to easily implement support for additional systems, as proposed in the future work section~\ref{sec:futurework}.
    Moreover it makes use of implementation-independent RESTful~\cite{rest} standard APIs.

    \item \emph{alleviate technical hindrances against its use} \\
    During the design and implementation phases of our solution, many considerations have been taken into account to alleviate technical constraints around its usage.
    \begin{enumerate}
        \item \emph{being easy to deploy, even in existing environments} \\
        Docker~\cite{merkel2014docker} is extensively used to both facilitate development and deployment as detailed in appendix~\ref{sec:docker}.
        A demo version of the whole system can be brought up with a single command.
        
        \item \emph{not degrading the user experience in existing systems} \\
        Modifications to existing code bases were done in a way that the existing features are not affected, this includes Glowing Bear, IRCT, i2b2, and MedCo.
        
        \item \emph{enforcing secure authentication} \\
        Implementation of the support for OpenID Connect, a modern standard for secure authentication, was done in all the components of our system.
        It was implemented from scratch for Glowing Bear, i2b2 and MedCo, and the existing implementation of IRCT was modified to fit our use-case.
        
        \item \emph{being open-source} \\
        Our system is open-source, not just by the licenses used, but also by our contributions to the community.
        All the modifications made to the existing i2b2, IRCT, Glowing Bear and MedCo were the subject of \emph{pull requests} on their original repositories, so that they can be considered and potentially mainlined into the original repositories.
        
        \item \emph{having a practical runtime} \\
        Query times are kept practical.
        When compared to the query times of the original systems, the overhead is minimal.
        In tranSMART's case it stays the same as the Glowing Bear implementation remains.
        In the i2b2 case there is the overhead of using IRCT, however this overhead is minimal and close to a constant as the only processing is the translation of the query. 
        This translation process is close to a simple mapping and is in the order of a dozen milliseconds (not including I/O).
        In the MedCo case the time is actually reduced, as the query do not pass through SHRINE and is sent directly to the MedCo nodes.
        
        
    \end{enumerate}


% part to say it will be fine with bigger query
% assumed qd, irct is kn, this gives n

% outside of translation, simple proxying


% % practical runtime
% Finally the query times are kept practical for users, i.e. the overhead compared to the original systems is kept minimal.
% This overhead is kept constant as the total runtime depends mainly on the time the clinical research systems take to answer to the query.
% The overhead is $O(c)$, $c=f(q)$ being linear to the number of query $f$ linear and $O(c) + O(a) = O(c + a) = O(c)$.
% For d fixed the query time is equivalent to .


% We make the hypothesis that the clinical research systems is 
% Assuming that the query times is function of $O(q)$, i.e. linear to the number of query terms, 

% Queries are constrsucted in  a way 
% it is extrapolated from exprimental measurements of experimentally determined by<medcopaper> that i2b2 query times are a linear function of the number of query terms $n$ times a factor proportional to the database size $d$: $O(nd)$.
% o(nd +n), o(n(d+1)) = o(n)


\end{enumerate}


% if time allows, measurements possibilities:
% - i2b2 query times vs picsure+i2b2 query times (overhead)
% - medco query times with pic-sure vs previously with shrine (refer to the paper)
% - compare gb on public i2b2 with i2b2 webclient public demo
% - irct: measure time it takes just this
 
 