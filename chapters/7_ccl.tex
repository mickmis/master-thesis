\chapter{Conclusion}
\label{sec:conclusion}

% review
Throughout this thesis we designed and implemented a system providing a front end able to query the clinical research systems \emph{i2b2}, \emph{tranSMART} and \emph{MedCo}, with negligible overhead compared to the original platforms.
The supported queries allow researchers to do cohort exploration based on common criteria, giving technical means to clinical sites to share their data.

%
By this we are alleviating the challenges stated in the introduction.
First the technical challenge, we are bringing together technically the two main open-source clinical research systems with an interoperability layer that can be used by a unique front end.
Second the legal and ethical challenge, by integrating MedCo in this interoperability layer, clinical sites are given the option to securely share their data with strong privacy and security guarantees.

All in all, with our system we took a step towards the technical convergence of the main clinical research systems.
We are providing researchers with a modern a powerful front end that abstracts away the technical differences between platforms, and enable clinical sites to share that would otherwise not be shared without the strong privacy and security guarantees offered. 



\section{Future Work}
\label{sec:futurework}

Future work on our system could be done on both the front end and the interoperability layer.
On the front end features that exists for tranSMART could be added for the PIC-SURE implementation, such as a standard mechanism for saving queries, data export, or advanced data analysis.
Glowing Bear and the interoperability layer could be easily extended to support more systems with low implementation effort, systems such as HAIL~\cite{hail}, SciDB~\cite{brown2010overview} or SHRINE~\cite{shrine}.
Those are examples of systems partially supported by PIC-SURE, but who need adaptations to be fully integrated.
To bring more uniformity and easier maintenance, the support of tranSMART could be brought through PIC-SURE.
We could also imagine a way to bring uniformity in the semantic data coming from clinical research systems, allowing to query data from multiple systems at a time.

A work already in progress on our solution consists in bringing the support of HAIL through Livy~\cite{livy} in IRCT, and combining it with i2b2 for powerful cohort exploration with genomic data.

% hail on spark, livy rest iface for spark

% todo below add info
% This can be done through the 

% todo citation HAIL / R / PIC-SURE: https://www.ncbi.nlm.nih.gov/pubmed/29267850

% write up for using hail and i2b2 at the same time more detailed, e.g. what would be said to denise 

% example: transmart and hail, because transmart not through irct

% 
