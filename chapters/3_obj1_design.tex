%Divide this into 2–3 sections
%-eg “Design/Architecture” and “Implementation” details
%-... but avoid generic section titles
%Start with high-level overview of solution (top-down)
%-Give the reader the bigger picture first
%-Figure with overview of system architecture works well
%-Roadmap helps as well
%Give examples and make them consistent (eg a running example)

\chapter{Interoperability Layer for Clinical Research Systems}

Interoperability Layer for Clinical Research Systems
Common Interoperability Components
If IRCT: description of sources
Connectors
i2b2
SHRINE
(tranSMART 16.2)
(tranSMART 17.1)

\section{Design of the Solution}

put the overall graph + two variants of transmart 17.1 with pic sure or not, if time allows
--> make those graphs


\section{Implementation of the Design}

step by step sectionning: incl. current status of software and modifications / implementations to bring in

1. irct setup + i2b2 + transmart 16.2 + shrine
deployment mainly, data? i2b2 demo data? fortransmart? 
DB setup for the systems
impl for shrine maybe? (extend the i2b2 one)

2. GB talks pic sure api
2.1. genericizes api talking within GB: keep compatibility with transmart rest v2: at the end working as before
2.2. implements talking to pic sure: at the end working with both pic sure + transmart rest v2 
-> keep compatiblity becasue that was the original discusion / try to get transmart 17.1 in irct but if not enough time this is as backup


- add support for shrine/medco with additional JAR packaged and put in wildfly
- add systems (by type) with pq-psql functions



%\subsection{Current Status of Softwares}
%\subsection{Glowing Bear}
%desc of how it works
