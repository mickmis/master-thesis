\chapter{High-Level System Design}
\label{sec:sysdesign}
%--> here not the descriptive part (should go to related work / background), but the analytic part
% only open source tech considered

In the open-source spirit, existing technologies and standard. Maximizing the value of the work. 
Integrating with existing systems.
In order to meet the requirements listed section~\ref{sec:requirements}, 


\section{Open-Source Technologies}
Technologies for this are naturally split in two categories, client (front end) and server (back end) sides.
In this section we compare and choose the basic building blocks of our solution and how they fit together from a high-level point-of-view.

\subsection{Back End Systems: Clinical Research Platforms}
As stated section~\ref{sec:requirements}, we target compatibility with the main open-source clinical research platforms. 
In that area the two main players are \emph{i2b2} and \emph{tranSMART}.
Less widespread, and building on \emph{i2b2}, also exists \emph{SHRINE} and \emph{MedCo}.

% § about i2b2: why support it

% § about tranSMART: why support it 

% § about MedCo: why support it: keep for obj. 2: \subsubsection{Privacy-Preserving Platform}

what exist there, which one we are targetting the compatibility for and why // descriptive part in related work / background
+ mention pic sure here or in separate subsec?

transmart-rest-api plugin: what is it? no api natively ?

checkout/mention more general systems: scidb / elasticsearch / olap-mdx (mdx is the query language, olap the model)-could be envisioned to be supported thorgh pic sure


\subsection{Front End System: Cohort Explorers}

cohort exploration
§ what's available (compatible with previously selected systems)
borderline by etriks (?)
GB
i2b2 webclient: outdated technologies
i2b2 workbench: heavy client, not portable enough
transmartApp 
todo: (description of the non relevant ones here? backgorund only things that we are actually using in the solution)

§ Why GB 
already compatible with 17.1

§ what does gb need to work (what is it using from the transmart rest api?)

§ mention building from scratch, but makes no sense

% summary
--> should conclude with what front end we want to use : GB

% we need to retain the current features of GB, users do no want change

\section{System Design Scenario}
Now that we chose the fundamental open-source building blocks of our solution, we must fit them together.
We have two natural approaches to do so: with or without a middle component.

there are two main approaches we can use to assemble them together.


Assembling these tools together can take several roads, present them here.
weigh the procs and cons
2 main models: with or wihtout a backend component
w/ backend component: can use IRCT as the backend
Alternative models? (OLAP/MDX investigations? To be discussed)
The final decision

--> should conclude that we want the scenario 2, i.e. with a backend

then paragraph to introduce the use of pic sure, which is an optional implementation option, but needs investigations -> next section

backend scenario> either implement or use sth existing


\subsection{PIC-SURE API \& IRCT Investigations}

% goal of investigations
Given the choice of the scenario that makes use of a backend, the option of using the IRCT as the backend for the system is to be evaluated.
On a first look it looks good 

% technical analysis
questions raised --> cf next §

% questions raised and answered
This first technical analysis raised some questions on IRCT, becasue we want the , we woul
Because using IRCT introduce a heavy dependency towards it for Glowing Bear, 


Decision on implementation scenario \& technologies
Description of 2 scenarios
PIC-SURE API Investigations
I2b2, SHRINE and tranSMART 16.2 current status of integration
Estimation of effort to integrate tranSMART 17.1


PIC-SURE API Investigations
§ what is being investigated, link to things explained before (why this is being investigated as potential solution)

Actions of irct on external defined by set of definitions (list: predicates supported, things executed on resources), resources declare what they support
Interface (RI) execute the action with the resource interface in the native protocol, result is returned as irct result

Architecture: 4 layers (the components): put diagram
Authentication done by irct (with external auth provider it seems?) - irct has set of credentials to talk to the resources 
It uses JWT for authentication -> not sure how this works
Jwt token is signed using the secret configured // https://auth0.com/docs/tokens/access-token
Openid connect // for tests:https://openidconnect.net/
---
, using maven/java
War deployed in wildfly or similar
Hibernate for auto-gen of db (postgresql seems to work, event if a ext thing is compatible only with oracle: hmlsssynonyms)
Sql files are provided to fill db (not tested yet)
Configuration: should be guessed (wildfly can’t deploy without it)
%Redirect_on_success: url to redirect after login
%Client_secret: shared secret with auth provider
%Userfield: user id field name in resp. From auth provider
%Client_id: client id when talking to auth provider?
%Domain: domain of auth provider?
%Keyoutinmuntes: time out of key (120 mins default)


analysis shoold be targeted: what is its goal? evaluate it meets the requirements of GB
§ analysis/description of the API: todo %http://bd2k-picsure.hms.harvard.edu/docs/IRCT_Protocol_1.0.pdf !!!!! (actually took the 1.1)
Notes:

Layer -> interoperability layer (check the api doc pdf)
introduce resource term

infos in wiki + github

§ questions that came up (list): from analyse of what’s available online, interrogations came up, include also the answers (check out list of questions that was made)

§ set up of the thing: compiling from sources etc. packaged with docker (links to that)

§ tests on the set up: what resources to take for tests? Must have is I2b2 and is supposed to work out the box (need shrine deployment then??) -- shrine not supported officially, but i2b2 
If it works w/ i2b2 -> good, it will work with shrine-medco, and transmart 17.1 as native GB support in worst-case scenario (nice to have: transmart 17.1 adapter w/ irct)

(conclusions on the investigations in subsections)


About the existing UI tools: 

--> should conclude here that we want to use pic sure

----------
\paragraph{tranSMART 17.1 Integration}
In order to consider using the PIC-SURE API, we need to ensure that 
§analyses of yes or not: Imoprtant: does GB matches with ability of the pic sure api? I.e. does the transmart api v2 can be replaced by pic sure 
Or not? Since irct supports i2b2/old transmart it should support transmart api v2

Irct works with resources: you ask the api about resource and what it can do, and you can query it with a unique api
Hard to say with the only doc for api there is, it s a bit brief (but should be fine because of i2b2-transmart support)
--> optional, best to have (also easier for GB developments)

§estimation of the effort needed



\section{Technical Choices Summary}
--> conclusion to resume the technical choices made


\paragraph{Front End}
\emph{Glowing Bear}, with a dedicated back end serving as an interoperability layer: \emph{IRCT}.

\paragraph{Supported Clinical Research Platforms}
The relevance of these systems in our case lie in their API and how they are used. 
To that extent, an exhaustive list of systems and API combinations we choose to support is the following:
\begin{itemize}
    \item tranSMART REST API v1 (versions >= 1.24 \& <= 16.2)~\cite{tranSMART-REST-API},
    \item tranSMART REST API v2 (versions >= 17.1)~\cite{tranSMART-REST-API},
    \item i2b2~\cite{i2b2-docs};
    \item SHRINE~\cite{todo}
    \item MedCo~\cite{todo}
\end{itemize}

Altogether these systems serve an enormous of data, and being able to access them from a unique front end would prove much beneficial to researchers.

% todo: GB -> heavy refactor -> explain and motivate