\chapter{APIs Usage Examples}

This appendix gives examples on how to use some APIs that are referred to in the thesis.

\section{tranSMART REST API v2}

% todo: make better and more
% todo: contains REAL examples (i.e. real response)
\subparagraph{List of studies}
The list of available studies is requested:
\begin{verbatim}
GET /v2/studies
\end{verbatim}

Response:
\begin{verbatim}
{
  "studies": 
  [
    {
      "id": -20,
      "studyId": "CATEGORICAL_VALUES",
      "bioExperimentId": -10,
      "dimensions": 
      [
        "concept",
        "patient",
        "study"
      ]
    },
    ...
}
\end{verbatim}

\subparagraph{Query Terms Tree}
The tree of the queryable terms is recuperated, with a depth of 2:
\begin{verbatim}
GET /v2/tree_nodes?root=&depth=2&tags=true
\end{verbatim}

Response:
\begin{verbatim}
{
  "tree_nodes":
    [
      {
        "name":"Vital Signs",
        "fullName":"\Vital Signs\",
        "name":"Vital Signs",
        "type":"UNKNOWN",
        "visualAttributes": ["FOLDER","ACTIVE"], 
        "children": [
          {
            "name":"Heart Rate",
            "fullName":"\Vital Signs\Heart Rate\",
            "conceptCode":"VSIGN:HR",
            "conceptPath":"\Vital Signs\Heart Rate\",
            "name":"Heart Rate",
            "type":"NUMERIC",
            "visualAttributes":["LEAF","ACTIVE","NUMERICAL"],
            "constraint": {
              "type": "concept",
              "conceptCode": "VSIGN:HR"
            }
          }
        ]
      },
      ...
    ]
}
\end{verbatim}

\subparagraph{Saved Queries}
The list of previously saved queries is retrieved:
\begin{verbatim}
GET /v2/queries
\end{verbatim}

Response:
\begin{verbatim}
{
  "id": 1,
  "name": "testquery",
  "patientsQuery": {
    "constraint": {
      "args": 
      [
        {
          "conceptCode": "birthdate",
          "conceptPath": "\Demographics\Birth Date\",
          "fullName": "\Projects\Survey 1\Demographics\Birth Date\",
          "name": "Birth Date",
          "type": "concept",
          "valueType": "DATE"
        },
        {
          "studyId": "SURVEY1",
          "type": "study_name"
        }
      ],
      "type": "and"
    },
    "dimension": "patient",
    "type": "subselection"
  },
  "observationsQuery": {
    "data": [ ]
  },
  "apiVersion": "2.0",
  "bookmarked": false,
  "createDate": "2018-03-11T15:30:07Z",
  "updateDate": "2018-03-11T15:30:07Z"
}
\end{verbatim}


\subparagraph{Query}

\begin{verbatim}
POST /v2/observations/counts_per_study_and_concept

{
  "constraint": {
    "type": "or",
    "args": [
      {
        "type": "subselection",
        "dimension": "patient",
        "constraint": {
          "type": "and",
          "args": [
            {
              "type": "concept",
              "conceptCode": "birthdate"
            },
            {
              "type": "study_name",
              "studyId": "SURVEY1"
            }
          ]
        }
      },
      {
        "type": "subselection",
        "dimension": "patient",
        "constraint": {
          "type": "and",
          "args": [
            {
              "type": "concept",
              "conceptCode": "O1KP:CAT1"
            },
            {
              "type": "study_name",
              "studyId": "ORACLE_1000_PATIENT"
            }
          ]
        }
      }
    ]
  }
}
\end{verbatim}


\section{PIC-SURE API}

\section{i2b2 API}
