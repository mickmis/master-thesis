\chapter{APIs Usage Examples}

% This appendix gives examples on how to use some APIs that are referred to in the thesis.

\section{JSON Web Token}
\label{sec:appendix-jwt}

\section{tranSMART REST API v2}

% todo: make better and more
% todo: contains REAL examples (i.e. real response)
% \paragraph{List of studies}
% The list of available studies is requested:
% \begin{verbatim}
% GET /v2/studies
% \end{verbatim}

% Response:
% \begin{verbatim}
% {
%   "studies": 
%   [
%     {
%       "id": -20,
%       "studyId": "CATEGORICAL_VALUES",
%       "bioExperimentId": -10,
%       "dimensions": 
%       [
%         "concept",
%         "patient",
%         "study"
%       ]
%     },
%     ...
% }
% \end{verbatim}

% \paragraph{Query Terms Tree}
% The tree of the queryable terms is recuperated, with a depth of 2:
% \begin{verbatim}
% GET /v2/tree_nodes?root=&depth=2&tags=true
% \end{verbatim}

% Response:
% \begin{verbatim}
% {
%   "tree_nodes":
%     [
%       {
%         "name":"Vital Signs",
%         "fullName":"\Vital Signs\",
%         "name":"Vital Signs",
%         "type":"UNKNOWN",
%         "visualAttributes": ["FOLDER","ACTIVE"], 
%         "children": [
%           {
%             "name":"Heart Rate",
%             "fullName":"\Vital Signs\Heart Rate\",
%             "conceptCode":"VSIGN:HR",
%             "conceptPath":"\Vital Signs\Heart Rate\",
%             "name":"Heart Rate",
%             "type":"NUMERIC",
%             "visualAttributes":["LEAF","ACTIVE","NUMERICAL"],
%             "constraint": {
%               "type": "concept",
%               "conceptCode": "VSIGN:HR"
%             }
%           }
%         ]
%       },
%       ...
%     ]
% }
% \end{verbatim}

% \paragraph{Saved Queries}
% The list of previously saved queries is retrieved:
% \begin{verbatim}
% GET /v2/queries
% \end{verbatim}

% Response:
% \begin{verbatim}
% {
%   "id": 1,
%   "name": "testquery",
%   "patientsQuery": {
%     "constraint": {
%       "args": 
%       [
%         {
%           "conceptCode": "birthdate",
%           "conceptPath": "\Demographics\Birth Date\",
%           "fullName": "\Projects\Survey 1\Demographics\Birth Date\",
%           "name": "Birth Date",
%           "type": "concept",
%           "valueType": "DATE"
%         },
%         {
%           "studyId": "SURVEY1",
%           "type": "study_name"
%         }
%       ],
%       "type": "and"
%     },
%     "dimension": "patient",
%     "type": "subselection"
%   },
%   "observationsQuery": {
%     "data": [ ]
%   },
%   "apiVersion": "2.0",
%   "bookmarked": false,
%   "createDate": "2018-03-11T15:30:07Z",
%   "updateDate": "2018-03-11T15:30:07Z"
% }
% \end{verbatim}


% \paragraph{Query}

% \begin{verbatim}
% POST /v2/observations/counts_per_study_and_concept

% {
%   "constraint": {
%     "type": "or",
%     "args": [
%       {
%         "type": "subselection",
%         "dimension": "patient",
%         "constraint": {
%           "type": "and",
%           "args": [
%             {
%               "type": "concept",
%               "conceptCode": "birthdate"
%             },
%             {
%               "type": "study_name",
%               "studyId": "SURVEY1"
%             }
%           ]
%         }
%       },
%       {
%         "type": "subselection",
%         "dimension": "patient",
%         "constraint": {
%           "type": "and",
%           "args": [
%             {
%               "type": "concept",
%               "conceptCode": "O1KP:CAT1"
%             },
%             {
%               "type": "study_name",
%               "studyId": "ORACLE_1000_PATIENT"
%             }
%           ]
%         }
%       }
%     ]
%   }
% }
% \end{verbatim}

% \subsection{Glowing Bear Workflow}
% calls:
% --- init step 0
% 1. /studies: list of studies, displayed under "Public Studies", some under "Private Studies" (distinction how??), and more, but how exactly are they mapped to the tree?

% 2. --> they are mapped with call to /tree\_nodes?root=\\ => each of the concept has a studyId to do the mapping
% --> call to /tree\_nodes has also parameters depth --> IRCT only has depth 1, tree traversal in UI might need modifs

% 3. /queries: list of saved queries returned

% 4. /jobs: export jobs (running or not) --> not sure how IRCT handles that -> investigate / might need a change

% 5. /counts\_per\_study\_and\_concept: return counts for all studies / sub things -> see if how to map to irct, either as data or as process?

% 6. /counts: with param true, to display the total counts available

% --- query step 1
% 7. /aggregates\_per\_concept : metadata of the concept, to display min max count etc. (aggregate data)
% triggered on drag n drop of concept
% --> use aggregate just as scidb again (so this should be a requirement of GB-or if not feature disabled)

% POST /v2/observations/aggregates\_per\_concept
% {constraint: {type: "concept", conceptCode: "O1KP:AGE"}}

% {
%   "aggregatesPerConcept": {
%     "O1KP:AGE": {
%       "numericalValueAggregates": {
%         "avg": 51.25,
%         "count": 1200,
%         "max": 65.0,
%         "min": 50.0,
%         "stdDev": 4.147509477069983
%       }
%     }
%   }
% }


% 8. /elements: param contraint: which concept // not sure of what the answer if for?
% GET /v2/dimensions/trial\%20visit/elements
% constraint:{"type":"concept","conceptCode":"O1KP:AGE"}

% {
%   "elements": 
%   [
%     {
%       "id": -101,
%       "relTimeLabel": "General",
%       "relTimeUnit": null,
%       "relTime": null
%     }
%   ]
% }


% starting here possible to change constraints \& concepts in UI, then click on update counts:
% 9. /counts\_per\_study\_and\_concept: update the counts according to the selected concepts-> return all relevant count (to update the tree) // constraint as param
% POST /v2/observations/counts\_per\_study\_and\_concept
% {
%   "constraint": {
%     "type": "subselection",
%     "dimension": "patient",
%     "constraint": {
%       "type": "and",
%       "args": [
%         {
%           "type": "and",
%           "args": [
%             {
%               "type": "concept",
%               "conceptCode": "O1KP:AGE"
%             },
%             {
%               "type": "value",
%               "valueType": "NUMERIC",
%               "operator": ">=",
%               "value": 55
%             }
%           ]
%         },
%         {
%           "type": "study_name",
%           "studyId": "ORACLE_1000_PATIENT"
%         }
%       ]
%     }
%   }
% }

% {
%   "countsPerStudy": {
%     "ORACLE_1000_PATIENT": {
%       "O1KP:AGE": {
%         "observationCount": 100,
%         "patientCount": 100
%       },
%       "O1KP:CAT1": {
%         "observationCount": 100,
%         "patientCount": 100
%       },
%       "O1KP:CAT10": {
%         "observationCount": 100,
%         "patientCount": 100
%       },
%       ......
%     }
%   }
% }



\section{PIC-SURE API}

% GET /resultService/available: returns the list of available results (of the user)
% returns the result id and the status
% <<< GB does first the call to get job list

% GET /resultService/resultStatus/ID
% if problem: message field (always try to get it, but can be null)
% if OK, fields are: result id, status, startime, endtime, data type

% GET /resultService/availableFormats/ID
% returns list of formats for a specific result

% GET /resultService/result/ID/format?download=yes
% download can be put to no (content type will change, set yes for download from browser with clickable link)

% \begin{verbatim}
% GET /resourceService/resources
% \end{verbatim}

% Response:
% \begin{verbatim}
% [
%   {
%     "id": 1,
%     "name": "resource name (e.g. i2b2-local)",
%     "implementation": "resource type (which implementation is used, e.g. i2b2XML)",
%     "relationships": [ supported relationships between tree nodes, e.g. CHILD ],
%     "logicaloperators": [ "AND", "OR", "NOT" ],
%     "predicates": [ supported predicates and their properties, e.g.: {
%         "predicateName": "predicate name (e.g. CONTAINS)",
%         "displayName": "displayed name (e.g. Contains)",
%         "description": "description",
%         "default": true if this predicate should be selected by default,
%         "fields": [ {
%             "name": "field name (e.g. By Encounter)",
%             "path": "field code (to be used in query, e.g. ENCOUNTER)",
%             "description": "By Encounter",
%             "required": true,
%             "dataTypes": [ data type of the value field(s) of this predicate ],
%             "permittedValues": [ permitted values, if it categorical ]
%           } ],
%         "dataTypes": [ data types to which this predicate applies (e.g. STRING) ],
%         "paths": [ paths to which this predicate applies (empty for all) ]
%       } ],
%     "selectOperations": [ supported operations for select (e.g. AGGREGATE) ],
%     "selectFields": [ supported fields for the operations, e.g. COUNT for AGGREGATE ],
%     "joins": [ ],
%     "sorts": [ ],
%     "processes": [ ],
%     "visualization": [ ],
%     "dataTypes": [ data types and their properties, e.g.: {
%         "name": "name of the data type",
%         "pattern": "regex to validate the value",
%         "description": "description of the data type"
%     } ]
%   },
%   ...
% ]
% \end{verbatim}


% The API request becomes:
% \begin{verbatim}
% GET /resourceService/path/<resource>/<path>/?relationship=<relationship>
% \end{verbatim}

% Response:
% \begin{verbatim}
% [
%   {
%     "pui": "/<resource>/<path>/",
%     "name": "Concept internal name",
%     "displayName": "Concept name",
%     "description": "Concept description",
%     "ontology": "Ontology Code",
%     "ontologyId": "Concept ID in the ontology",
%     "relationships": [ supported relationships ],
%     "counts": {},
%     "dataType": {
%       "name": "Data type name",
%       "pattern": "Validation Regex",
%       "description": "Data type description"
%     },
%     "attributes": {
%       "visualattributes": "Visual attributes",
%       "customAttributeName": "customAttributeValue"
%     }
%   },
%   ...
% ]
% \end{verbatim}


\section{i2b2 API}
\label{sec:appendix-i2b2}


% todo: JWT example + say that when exchanged they are always base64 encoded

% todo: i2b2 and medco pic-sure resources?
